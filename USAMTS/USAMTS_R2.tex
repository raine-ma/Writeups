%%%%%%%%%%%%%%%%%%%%%%%%%%%%%%%%%%%%%%%%%%%%%%
%%                                          %%
%% USE THIS FILE TO SUBMIT YOUR SOLUTIONS   %%
%%                                          %%
%% You must have the usamts.tex file in     %%
%% the same directory as this file.         %%
%% You do NOT need to submit this file or   %%
%% usamts.tex with your solutions.  You     %%
%% only need to submit the output PDF file. %%
%%                                          %%
%% I ALTERED THE FILE usamts.tex            %%
%%                                          %%
%% If you have any questions or problems    %%
%% using this file, or with LaTeX in        %%
%% general, please go to the LaTeX          %%
%% forum on the Art Of Problem Solving      %%
%% web site, and post your problem.         %%
%%                                          %%
%%%%%%%%%%%%%%%%%%%%%%%%%%%%%%%%%%%%%%%%%%%%%%

%%%%%%%%%%%%%%%%%%%%%%%%%%%%%%%%%%%%%%%%%%%%
%% I ALTERED THE FOLLOWING LINES
\documentclass[11pt, letterpaper]{article}
\usepackage{amsmath,amssymb,amsthm}
\usepackage[pdftex]{graphicx}
\usepackage{fancyhdr}
\usepackage{listings}
\pagestyle{fancy}
\usepackage{setspace}
\usepackage{minted}
\setstretch{1.5}
\usepackage{listings}
\begin{document}
\include{usamts}
%% I ALTERED THE ABOVE LINES
%%%%%%%%%%%%%%%%%%%%%%%%%%%%%%%%%%%%%%%%%%%%


%% If you would like to use Asymptote within this document (which is optional), 
%% you can find out how at the following URL:
%%
%%   http://www.artofproblemsolving.com/Wiki/index.php/Asymptote:_Advanced_Configuration
%%
%% As explained there, you will want to uncomment the line below.  But be
%% sure to check the website because there are several other steps that must 
%% be followed.
%% \usepackage{asymptote}

%% Enter your real name here
%% Example: \realname{David Patrick}
\realname{Raine Ma}

%% Enter your USAMTS username here
%% Example: \username{DPatrick}
%% IMPORTANT
%% If your username contains one of the following characters:
%%      # $ ~ _ ^ % { } &
%% then this character must be preceded by a backslash \
%% for example: if your user name is math_genius, then the line below should 
%% \username{math\_genius}

\username{433mea}


\usamtsid{43191}

\usamtsyear{36}
\usamtsround{2}
\begin{solution}{1} 
    Solution is shown below:
    \begin{center}
    \includegraphics[width=1.2\linewidth]{problem_1_2.png}
    \end{center}

\end{solution}


\begin{solution}{2}

Examining the problem, we notice that it would likely take very complex casework and proofs to solve this problem. Also, I could potentially leave out many cases due to human error. However, since use of computer programs is allowed on this contest, I wrote a Python program to exhaustively search all possible grids. The program utilizes backtracking, which is a well-known algorithm used in constraint satisfaction problems such as this one. 

\begin{minted}[mathescape,
               linenos,
               numbersep=5pt,
               gobble=2,
               frame=lines,
               framesep=2mm]{python}
    from time import time

    start = time()
    
    """
    I used
    Python 3.11.2 (v3.11.2:878ead1ac1, Feb  7 2023, 10:02:41) 
    [Clang 13.0.0 (clang-1300.0.29.30)] on darwin
    """
    
    def isMagic(grid, target):
        #Checks whether grid is magic square
        lis = [sum(grid[0:3]),sum(grid[3:6]),sum(grid[6:9]),
               sum(grid[::3]),sum(grid[1::3]),sum(grid[2::3]),
               sum(grid[::4]),grid[2]+grid[4]+grid[6]]
        return len(set(lis)) == 1 and lis[0] == target
    
    def toStr(grid):
        return 'skibidi'.join(map(str, grid))
        #creates unique representation for each grid
        #since lists are unhashable
    
    def atMost8(grid):
        return len(set(grid)) < 9
        #Checks whether at most 8 distinct values in grid
    
    def rotate(grid): #Rotate 90 clockwise
        return [grid[6], grid[3], grid[0],
                grid[7], grid[4], grid[1],
                grid[8], grid[5], grid[2]]
    
    def reflect(grid): #switch cols 1 and 3
        return [grid[2], grid[1], grid[0],
                grid[5], grid[4], grid[3],
                grid[8], grid[7], grid[6]]
    
    def syms(grid): #Generates all possible symmetric variations of grid
        s = []
        current = grid
        for i in range(4):
            s.append(current)
            s.append(reflect(current))
            current = rotate(current)
        return s
    
    def search(ind, left, grid, dist_vals, target, sols, seen):
        if ind == len(left): #If grid is filled already
            if isMagic(grid, target): 
                #appends solution if it has not been discovered
                symmetries = syms(grid)
                s = min(toStr(sym) for sym in symmetries)
                #Distinct string representation for each grid
                #Ensures we are not overcounting
                if s not in seen:
                    seen.add(s)
                    sols.append(grid[:])
            return
        #else, if grid is not filled keep going
        pos = left[ind]
        for val in range(1, 13): #Iterates 1-12
            if val not in dist_vals or len(dist_vals) < 8:
            #Checking whether value is distinct in grid
                #Whether adding another distinct value would violate at most 8
                grid[pos] = val #If it can be added set current pos to val
                new = val not in dist_vals #If val is not previously used in grid
                if new: #Add to distinct values tracker if new
                    dist_vals.add(val)
    
                row, col = pos // 3, pos % 3
                #bools that check whethers any constraints set by problem are violated
                #this check before helps reduce the amount of times the code runs
                #will save some exexcution time
                bool1 = (all(grid[3 * row + c] != 0 for c in range(3)) and 
                         sum(grid[3 * row + c] for c in range(3)) != target)
                bool2 = (all(grid[r * 3 + col] != 0 for r in range(3)) 
                         and sum(grid[r * 3 + col] for r in range(3)) != target)
                #Checks rows and columns
                bool3 = (pos in [0, 4, 8] and all(grid[i] != 0 for i in [0, 4, 8]) 
                         and sum(grid[i] for i in [0, 4, 8]) != target)
                bool4 = (pos in [2, 4, 6] and all(grid[i] != 0 for i in [2, 4, 6]) and 
                         sum(grid[i] for i in [2, 4, 6]) != target)
                #checks diagonals
                valid = not bool1 and not bool2 and not bool3 and not bool4 
                #whether grid satisfies conditions
    
                if valid: #Moves on to next position if grid is valid
                    search(ind + 1, left, grid, dist_vals, target, sols, seen)
        
                if new: #Remove if it is new
                    dist_vals.remove(val)
                grid[pos] = 0 #reset for next iteration
    
    sols = []
    seen = set()
    
    for target in range(3,37): 
        """
        Sum of integers 1-12 = 108, so strong upper bound
        on sums not exceeding 108/3 = 36.
        If grid contains all 1s, sum is 9, so strong lower
        bound of sum being greater than or equal to 3.
        """
        for center in range(1, 13):
            #Grid can contain ints 1-12, center can be 1-12
            grid = [0] * 9
            grid[4] = center #set center
    
            blanks = [i for i in range(9)]
            #Iterate based on positions of 1 and 2
            #assign 1 and 2 to distinct positions
            for i in range(len(blanks)):
                for j in range(i + 1, len(blanks)):
                    for pair in [(1, 2), (2, 1)]:
                        temp = grid[:]
                        temp[blanks[i]] = pair[0]#Check all possible positions of 1 and 2 
                        temp[blanks[j]] = pair[1]
                        #initialize positions left
                        left = [p for p in blanks if p not in [blanks[i], blanks[j]]]
                        dist_vals = {1, 2, center} 
                        #tracks how many distinct vals left
                        #begin backtracking from index 0
                        search(0, left, temp, dist_vals, target, sols, seen)
    
    for sol in sols:
        if not atMost8(sol):
            sols = list(filter((sol).__ne__, sols))
            #Removes sols with more than 8 distinct values
    
    for sol in sols:
        print(sol) #prints distinct solutions
    
    print("The program took " + str(time()-start) + " seconds") #runtime check (for fun)
    \end{minted}
\newpage
Note that the program exhaustively tries every single possible grid, calculating all 5 sums for each. The program also ensures no grid has its rotations and reflections overcounted. It also filters through solutions to ensure they only have at most eight distinct values. Therefore, this program has searched through all possible grids. The output is shown below:

\begin{minted}[mathescape,
               linenos,
               numbersep=5pt,
               gobble=2,
               frame=lines,
               framesep=2mm]{csharp}
    [2, 1, 3, 3, 2, 1, 1, 3, 2]
    [2, 3, 4, 5, 3, 1, 2, 3, 4]
    [2, 5, 5, 7, 4, 1, 3, 3, 6]
    The program took 104.74400997161865 seconds
\end{minted}

Therefore, there are \boxed{3} distinct solutions, namely:

$$\begin{bmatrix}
2 & 1 & 3\\
3 & 2 & 1\\
1 & 3 & 2
\end{bmatrix}$$


$$\begin{bmatrix}
2 & 3 & 4\\
5 & 3 & 1\\
2 & 3 & 4
\end{bmatrix}$$


$$\begin{bmatrix}
2 & 5 & 5\\
7 & 4 & 1\\
3 & 3 & 6
\end{bmatrix}$$

\end{solution}

\begin{solution}{3}

\includegraphics[width=1\linewidth]{iff.png}
We first prove "If $\overline{AB}$ is perpendicular to $\overline{CE}$, then $\triangle EPQ$ is equilateral." Choose $D$ arbitrarily.\\

First, let us begin angle chasing. Let $m\angle DBA$ = $\theta$. By definition of a circumcircle, $m\angle DPA$ = $2\theta$.  By triangle angle sum, $m\angle BEP$ = $90-\theta$. $\overline{\rm AE} = \overline{\rm EB}$, and $m\angle AFE = m\angle BEF = 90^\circ$, so $m\angle AEP$ = $m\angle BEP = 90-\theta$ by the perpendicular bisector theorem and isosceles triangle theorem. $m\angle AEP + m<DEA = 90 + \theta$, therefore $m\angle DEA = 2\theta$. Notice $$m\angle DEA = m\angle DPA$$ Therefore, $ADEP$ is cyclic. Since $\triangle DEA$ is inside $ADEP$, it follows that the center of $ADEP$ is $Q$, since $Q$ is the circumcenter of $\triangle DEA$. Since $P$ and $E$ both lie on a circle with center $Q$, $\overline{\rm PQ} = \overline{\rm QE}$. By the isosceles triangle theorem, $m\angle QPE = m\angle QEP$. \\

Observe that both $P$ and $Q$ are equidistant from both $A$ and $D$ by definition of a circumcenter. By the converse of the perpendicular bisector theorem, both $P$ and $Q$ must lie on the perpendicular bisector of $\overline{\rm AD}$. Note that the perpendicular bisector of $\overline{\rm AD}$ is parallel to $\overleftrightarrow{BH}$, since they both form $90^\circ$ angles with $\overleftrightarrow{AC}$. Therefore, $\overleftrightarrow{PQ} \parallel \overleftrightarrow{BH}$.\\ 

$P$ lies on $\overline{CF}$, as it must be equidistant from both $A$ and $B$. Since $CE \perp AB$, E lies on $CF$. Therefore, $\overleftrightarrow{PE} \parallel \overleftrightarrow{CF}$. Since $\overleftrightarrow{BH}$ makes a $60^\circ$ angle with $\overleftrightarrow{CF}$ it follows that $\overleftrightarrow{PQ}$ also makes a $60^\circ$ angle with $\overleftrightarrow{PE}$. By this, $m\angle QPE$ = $60^\circ$. Therefore, $m\angle QPE$ = $m\angle QEP$ = $60^\circ$. By triangle angle sum, $m\angle QPE$ = $m\angle QEP$ = $m\angle EQP$ = $60^\circ$. Therefore, $\triangle EPQ$ is equilateral.\\

\includegraphics[width=1\linewidth]{onlyif.png}
Now we prove: "If $\overleftrightarrow{AB}$ is not perpendicular to $\overleftrightarrow{CE}$, then $\triangle EPQ$ is not equilateral."

Consider $\angle QPE$. We have proved that $\overleftrightarrow{PQ} \parallel \overleftrightarrow{BH}$, by the definition of a circumcenter. For $\overleftrightarrow{PE}$ to make a $60^\circ$ angle with $\overleftrightarrow{QP}$, it must be that $\overleftrightarrow{PE} \parallel \overleftrightarrow{\rm CG}$. $P$ lies on $\overleftrightarrow{CG}$ by definition of a circumcenter. Therefore, $\overleftrightarrow{PE} \parallel \overleftrightarrow{CG}$ if and only if $E$ lies on $\overline{\rm CG}$. If $E$ was on $\overleftrightarrow{CG}$, $\overline{CE} \perp \overline{AB}$, violating our assumption that $\overline{AB} \not\perp \overline{CE}$. Since $PE \not\parallel CG$, this means that $\overleftrightarrow{PE}$ will form some angle other than $60^\circ$ with $\overleftrightarrow{PQ}$. Therefore, $m\angle QPE \neq 60^\circ $ in this case. Therefore, if $\overline{\rm AB} \not\perp \overline{\rm CD}$, $\triangle EPQ$ is not an equilateral triangle. We conclude that $\triangle EPQ$ is equilateral triangle if and only if $\overline{\rm AB} \perp \overline{\rm CD}$.

\if
Define $A = (-1,0), B = (1,0), C = (0,\sqrt{3})$. Without loss of generality, let D remain at the same point. 

Define R to be the circumcenter of APD. We note that P does not shift when we shift E, as ABD remains constant. Recall Q is the circumcenter of AED. 
Note that there is exactly one point E that is the same distance as P away from R while being at the same angle 60 from P. Without loss of generality, define BD = ax. Note that when AB is perpendicular to CE,  CE is x = 0. Therefore, E is (0,a) for some constant a. 

Now when AB is not perpendicular to CE, define CE = bx + c, where b neq 0. Since E is the intersection of CE and BD, we solve and obtain x = c/(a-b). x equals zero if and only if c equals zero. However, if c = 0 and b neq 0, it would be impossible for  can pass through (0,sqrt(3)) and (0,a)

Since this point E occurs when AB is perpendicular to CE, it must not occur otherwise. 
\fi
\end{solution}

\begin{solution}{4}
\iffalse
Consider $x_1$; $\frac{1}{2}$ of all subsets will contain $x_1$. If $x_1 \in T$, it will be the first element. Therefore, the expected contribution of $x_1$ is $$\frac{1}{2}\cdot x_1$$. 

Consider $x_2$; $\frac{1}{2}$ of all subsets will contain $x_2$. If $x_2 \in T$, it will be the either the first or second element. If it is the first element, $x_1 \notin T$, being the second element if $x_1 \in T$. Therefore, the expected contribution of $x_2$ is $$\frac{1}{2}\cdot\frac{1}{2}\cdot x_2 + \frac{1}{2}\cdot\frac{1}{2}\cdot-2\cdot x_2 = -\frac{1}{4}\cdot x_2$$.
\fi
Observe that for $x_i$ to occupy position $k$ in $T$, exactly $k-1$ elements from ${x_1,x_2,\dotsc,x_{i-1}}$ must precede $x_k$. There are ${i-1 \choose k-1}$ ways to select these elements. For the other ${n-i}$ elements, they do not affect the position, but we must consider them. Hence, the probability that $x_i$ occupies position $k$ is $$\frac{{i-1 \choose k-1} \cdot 2^{n-i}}{2^n} = \frac{{i-1 \choose k-1}}{2^i}$$

This holds for all $i \neq 1$, we will consider this case later. Denote $E[x_i]$ the expected contribution of $x_i$, where $i \neq 1$. Then $$E[x_i] = \sum_{k=1}^{i}(-1)^{k+1}\cdot k \cdot x_i \cdot \frac{{i-1 \choose k-1}}{2^i}$$

Define the inner sum $$S_i = \sum_{k=1}^{i}(-1)^{k+1}\cdot k \cdot {i-1 \choose k-1}$$

Then $$E[x_i] = \frac{x_i}{2^i}\cdot S_i$$

We can rewrite $S_i$ as 

$$S_i = \sum_{k=1}^{i}(-1)^{k+1}\cdot {(i-1)} \cdot {i-2 \choose k-2}$$

Let $j = k - 2$. $\binom{i-2}{j} = 0$ for $j < 0$, so the sum effectively starts at $j = 0$. 

$$S_i = -{(i-1)}\sum_{j=0}^{i-2}(-1)^{j} \cdot {i-2 \choose j}$$

By the binomial theorem, this equals $$-{(i-1)}\cdot(1-1)^{i-2} = -{(i-1)}\cdot0^{i-2}$$ This evaluates to $-1$ for $i = 2$, and $0$ for all $i > 2$. Substituting $S_i$ back into $E[x_i]$, we obtain $$E[x_i] = x_i \cdot \frac{\begin{cases} 
      0 &x > 2 \\
      -1 &  x = 2 
   \end{cases}}{2^i}
$$

Using this formula, we see that $x_2$ has an expected contribution of $-\frac{x_2}{4}$ and all $x_i$ where $i > 2$ have a contribution of zero. We now consider $x_1$. Note that $x_1$ will appear in exactly half of the subsets, and when it does, it will always be the first element. Therefore, the expected contribution of $x_1$ is $\frac{x_1}{2}$. Summing all cases, the expected value of $T$ is $$\boxed{\frac{x_1}{2} - \frac{x_2}{4}}$$.

\end{solution}

\begin{solution}{5}

Define $$\alpha = \sqrt[3]{5} + \sqrt[3]{25}$$

Observe that the following equation holds: 

$$\alpha^3 = 15\alpha +30$$ 

We prove that any powers of $\alpha$ greater than or equal to 3 can be expressed in terms of $1,\alpha$, and $\alpha^2$. We proceed with induction. Assume that  $$z\cdot\alpha^k = b_0 + b_1\alpha+b_2\alpha^2$$ for some $k \ge 0$, where $z,b_0, b_1,b_2 \in \mathbb{Z}$. Then $$z\cdot\alpha^{k+1} = b_0\alpha + b_1\alpha^2+b_2\alpha^3$$ Substitute $\alpha^3 = 15\alpha +30$ and simplify to obtain $$z\cdot\alpha^{k+1} = 30b_2 + (b_0+15b_2)\alpha + b_1\alpha^2$$ Therefore, if $z\cdot a^k$ can be expressed as $b_0 + b_1\alpha+b_2\alpha^2$, where  $z,b_0, b_1,b_2,\in \mathbb{Z}$, then $z\cdot a^{k+1}$ can as well. Since the base case $k=0$ holds, this result holds for all $z,k$ we are interested in.\\

We now prove that $2\sqrt[3]{5} + 3\sqrt[3]{25}$ cannot be expressed in terms of $\alpha^0,\alpha^1$, or $\alpha^2$ with integer coefficients. For the sake of contradiction, assume that some $c_0,c_1,c_2 \in \mathbb{Z}$ satisfy $$2\sqrt[3]{5} + 3\sqrt[3]{25} = c_0 + c_1\alpha+c_2\alpha^2$$

Substituting $\alpha = \sqrt[3]{5} + \sqrt[3]{25}$, we obtain $$c_0 + c_1\alpha+c_2\alpha^2 = c_0 + c_1\sqrt[3]{5} + c_1\sqrt[3]{25}+ c_2\sqrt[3]{25} + 10c_2 + 5c_2\sqrt[3]{5}$$ Group like terms: $$(c_0 +10c_2) + (c_1+5c_2)\sqrt[3]{5} + (c_1+c_2)\sqrt[3]{25} = 2\sqrt[3]{5} + 3\sqrt[3]{25}$$We obtain a system of equations $$\begin{cases} 
      (1) \quad c_0+10c_2 =0\\
      (2) \quad c_1+5c_2=2\\
      (3) \quad c_1+c_2=3
   \end{cases}$$

From equation (3): $$c_1+c_2 = 3 \implies c_1 = 3-c_2$$

Substitute this result into equation (2):

$$3 + 4c_2 = 2$$ We find that $c_2 = -\frac{1}{4}$, which is not an integer. This violates our assumption that $c_0 , c_1,$ and $c_2$ are integers. Therefore, $2\sqrt[3]{5} + 3\sqrt[3]{25}$ cannot be expressed in terms of $c_0 + c_1\alpha+c_2\alpha^2$, where $c_0,c_1,c_2 \in \mathbb{Z}$. Our previous result found that any $z\cdot\alpha^k$ can be expressed in terms of $1,\alpha,\alpha^2$, and therefore any sum of $z\cdot\alpha^k$, where $z\in\mathbb{Z}$ and $k \in \mathbb{N}$. Therefore,  $2\sqrt[3]{5} + 3\sqrt[3]{25}$ cannot be expressed as a sum of integer multiples of powers of $\alpha$, meaning that no polynomial with integer coefficients exists such that $P(\alpha) = 2\sqrt[3]{5} + 3\sqrt[3]{25}$.

\end{solution}



\end{document}
